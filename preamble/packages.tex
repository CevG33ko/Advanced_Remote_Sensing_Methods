% Document class

\documentclass[
	draft=false, % shows little squares at the end of a overfull line. In case with incompatibility with other packages use: overfullrule=true to show these lines.
	oneside, % oneside or twoside
	a4paper, % din a4 format
	titlepage=firstiscover, % first page is a cover
	headsepline=true, % line between header and text
	footsepline=false, % line between text and footer number
	fontsize=12pt, % font size
	parskip=false, % adds a space at the beginning of a line after \par
	footnotes=nomultiple, % mulitiple or nomultiple, manages the devider between two footnote numbers, multiple doesn't work with hyperref
]{scrreprt}

% for labeling list
\setkomafont{labelinglabel}{\bfseries\sffamily}
\setkomafont{labelingseparator}{\normalfont}

\usepackage{fontspec}
    \setmainfont{DejaVuSerifCondensed}
    \setsansfont{DejaVuSansCondensed}
    \setmonofont{JetBrains Mono Regular Nerd Font Complete Mono}

% to work with pictures
\usepackage{graphicx}
\graphicspath{{./img/}}
\DeclareGraphicsExtensions{.pdf,.png,.jpg} % prio1 = pdf, prio2 = png

% to wrap figures with text
\usepackage{wrapfig}

% Documentinformations
% Hyperlink creations and PDF Informations
\usepackage[
	draft=false,
	colorlinks=true, % use colors instead of frames
]{hyperref}

% For a good german
\usepackage[utf8]{inputenc}
\usepackage[english]{babel} % or german

% Bibliography
\bibliographystyle{unsrt}

% for better math and more symbols
\usepackage{amsmath}
\usepackage{amssymb}
\usepackage[version-1-compatibility]{siunitx} % compatibility because you will find a lot of v1 hints with google

% to avoid some error because siunitx uses it's own fonts
\usepackage{textcomp}

% for source code
\usepackage{listings}
\usepackage{color}


\definecolor{mygreen}{rgb}{0,0.6,0}
\definecolor{mygray}{rgb}{0.5,0.5,0.5}
\definecolor{myCodeBackground}{rgb}{0.98,0.98,0.98}
\definecolor{mymauve}{rgb}{0.58,0,0.82}
\definecolor{myAmethyst}{rgb}{0.6, 0.4, 0.8}
\definecolor{myCadmiumOrange}{rgb}{0.93, 0.53, 0.18}

\lstset{
backgroundcolor=\color{myCodeBackground},   % choose the background color; you must add \usepackage{color} or \usepackage{xcolor}; should come as last argument
	basicstyle=\footnotesize\ttfamily,        % the size of the fonts that are used for the code
	breakatwhitespace=false,         % sets if automatic breaks should only happen at whitespace
	breaklines=true,                 % sets automatic line breaking
	captionpos=b,                    % sets the caption-position to bottom
	deletekeywords={},            % if you want to delete keywords from the given language
	escapeinside={\%<LaTeX>}{</LaTeX>},          % if you want to add LaTeX within your code
	extendedchars=true,              % lets you use non-ASCII characters; for 8-bits encodings only, does not work with UTF-8
	firstnumber=1,                % start line enumeration with line 1000
	frame=none,	                   % adds a frame around the code
	keepspaces=true,                 % keeps spaces in text, useful for keeping indentation of code (possibly needs columns=flexible)
	language=C++,                 % the language of the code
	numbers=left,                    % where to put the line-numbers; possible values are (none, left, right)
	numbersep=5pt,                   % how far the line-numbers are from the code
	rulecolor=\color{green},         % if not set, the frame-color may be changed on line-breaks within not-black text (e.g. comments (green here))
	showspaces=false,                % show spaces everywhere adding particular underscores; it overrides 'showstringspaces'
	showstringspaces=false,          % underline spaces within strings only
	showtabs=false,                  % show tabs within strings adding particular underscores
	stepnumber=2,                    % the step between two line-numbers. If it's 1, each line will be numbered
	tabsize=2,	                   % sets default tabsize to 2 spaces
	title=\lstname,                   % show the filename of files included with \lstinputlisting; also try caption instead of title
	keywordstyle=\bfseries\color{mygreen},       % keyword style
	morekeywords={},            % if you want to add more keywords to the set
	numberstyle=\tiny\color{black}, % the style that is used for the line-numbers
	stringstyle=\color{myCadmiumOrange},     % string literal style
	commentstyle=\itshape\color{myAmethyst},    % comment style
	identifierstyle=\color{blue},
}

% for the acronym page
\usepackage[printonlyused]{acronym}   % \ac
